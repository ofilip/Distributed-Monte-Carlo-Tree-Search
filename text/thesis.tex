%%% Hlavní soubor. Zde se definují základní parametry a odkazuje se na ostatní části. %%%

%% Verze pro jednostranný tisk:
% Okraje: levý 40mm, pravý 25mm, horní a dolní 25mm
% (ale pozor, LaTeX si sám přidává 1in)
\documentclass[12pt,a4paper]{report}
\setlength\textwidth{145mm}
\setlength\textheight{247mm}
\setlength\oddsidemargin{15mm}
\setlength\evensidemargin{15mm}
\setlength\topmargin{0mm}
\setlength\headsep{0mm}
\setlength\headheight{0mm}
% \openright zařídí, aby následující text začínal na pravé straně knihy
\let\openright=\clearpage

%% Pokud tiskneme oboustranně:
% \documentclass[12pt,a4paper,twoside,openright]{report}
% \setlength\textwidth{145mm}
% \setlength\textheight{247mm}
% \setlength\oddsidemargin{15mm}
% \setlength\evensidemargin{0mm}
% \setlength\topmargin{0mm}
% \setlength\headsep{0mm}
% \setlength\headheight{0mm}
% \let\openright=\cleardoublepage

%% Použité kódování znaků: obvykle latin2, cp1250 nebo utf8:
\usepackage[utf8]{inputenc}

%% Ostatní balíčky
\usepackage{graphicx}
\usepackage{amsthm}


%% Balíček hyperref, kterým jdou vyrábět klikací odkazy v PDF,
%% ale hlavně ho používáme k uložení metadat do PDF (včetně obsahu).
%% POZOR, nezapomeňte vyplnit jméno práce a autora.
\usepackage[ps2pdf,unicode]{hyperref}   % Musí být za všemi ostatními balíčky
\hypersetup{pdftitle=Název práce}
\hypersetup{pdfauthor=Jméno Příjmení}

\usepackage{apacite}

%%% Drobné úpravy stylu

% Tato makra přesvědčují mírně ošklivým trikem LaTeX, aby hlavičky kapitol
% sázel příčetněji a nevynechával nad nimi spoustu místa. Směle ignorujte.
\makeatletter
\def\@makechapterhead#1{
  {\parindent \z@ \raggedright \normalfont
   \Huge\bfseries \thechapter. #1
   \par\nobreak
   \vskip 20\p@
}}
\def\@makeschapterhead#1{
  {\parindent \z@ \raggedright \normalfont
   \Huge\bfseries #1
   \par\nobreak
   \vskip 20\p@
}}
\makeatother

% Toto makro definuje kapitolu, která není očíslovaná, ale je uvedena v obsahu.
\def\chapwithtoc#1{
\chapter*{#1}
\addcontentsline{toc}{chapter}{#1}
}

\begin{document}

% Trochu volnější nastavení dělení slov, než je default.
\lefthyphenmin=2
\righthyphenmin=2

%%% Titulní strana práce

\pagestyle{empty}
\begin{center}

\large

Charles University in Prague

\medskip

Faculty of Mathematics and Physics

\vfill

{\bf\Large MASTER THESIS}

\vfill

\centerline{\mbox{\includegraphics[width=60mm]{img/logo.eps}}}

\vfill
\vspace{5mm}

{\LARGE Bc. Ondřej Filip}

\vspace{15mm}

% Název práce přesně podle zadání
{\LARGE\bfseries Distributed Monte-Carlo Tree Search for Games with Team of Cooperative Agents}

\vfill

% Název katedry nebo ústavu, kde byla práce oficiálně zadána
% (dle Organizační struktury MFF UK)
Department of Theoretical Computer Science and Mathematical Logic

\vfill

\begin{tabular}{rl}

Supervisor of the master thesis: & Mgr. Viliam Lisý, MSc. \\
\noalign{\vspace{2mm}}
Study programme: & Theoretical Computer Science \\
\noalign{\vspace{2mm}}
Specialization: & Neprocedural Programming and Artifitial Inteligence \\
\end{tabular}

\vfill

% Zde doplňte rok
Prague 2013

\end{center}

\newpage

%%% Následuje vevázaný list -- kopie podepsaného "Zadání diplomové práce".
%%% Toto zadání NENÍ součástí elektronické verze práce, nescanovat.

%%% Na tomto místě mohou být napsána případná poděkování (vedoucímu práce,
%%% konzultantovi, tomu, kdo zapůjčil software, literaturu apod.)

\openright

\noindent
Dedication.

\newpage

%%% Strana s čestným prohlášením k diplomové práci

\vglue 0pt plus 1fill

\noindent
I declare that I carried out this master thesis independently, and only with the cited
sources, literature and other professional sources.

\medskip\noindent
I understand that my work relates to the rights and obligations under the Act No.
121/2000 Coll., the Copyright Act, as amended, in particular the fact that the Charles
University in Prague has the right to conclude a license agreement on the use of this
work as a school work pursuant to Section 60 paragraph 1 of the Copyright Act.

\vspace{10mm}

\hbox{\hbox to 0.5\hsize{%
In ........ date ............
\hss}\hbox to 0.5\hsize{%
signature of the author
\hss}}

\vspace{20mm}
\newpage

%%% Povinná informační strana diplomové práce

\vbox to 0.5\vsize{
\setlength\parindent{0mm}
\setlength\parskip{5mm}

Název práce:
Distribuovaný Monte-Carlo Tree Search pro hry s týmem kooperujících agentů
% přesně dle zadání

Autor:
Bc. Ondřej Filip

Katedra:  % Případně Ústav:
Katedra teoretické informatiky a matematické logiky
% dle Organizační struktury MFF UK

Vedoucí diplomové práce:
Mgr. Viliam Lisý, MSc., Centrum agentních technologií, České Vysoké Učení Technické v Praze
% dle Organizační struktury MFF UK, případně plný název pracoviště mimo MFF UK

Abstrakt:
% abstrakt v rozsahu 80-200 slov; nejedná se však o opis zadání diplomové práce

Klíčová slova:
% 3 až 5 klíčových slov

\vss}\nobreak\vbox to 0.49\vsize{
\setlength\parindent{0mm}
\setlength\parskip{5mm}

Title:
Distributed Monte-Carlo Tree Search for Games with Team of Cooperative Agents
% přesný překlad názvu práce v angličtině

Author:
Bc. Ondřej Filip

Department:
Department of Theoretical Computer Scientce and Mathematical Logic
% dle Organizační struktury MFF UK v angličtině

Supervisor:
Mgr. Viliam Lisý, MSc., Agent Technology Center, Czech Technical University in Prague
% dle Organizační struktury MFF UK, případně plný název pracoviště
% mimo MFF UK v angličtině

Abstract:
% abstrakt v rozsahu 80-200 slov v angličtině; nejedná se však o překlad
% zadání diplomové práce

Keywords:
% 3 až 5 klíčových slov v angličtině

\vss}

\newpage

%%% Strana s automaticky generovaným obsahem diplomové práce. U matematických
%%% prací je přípustné, aby seznam tabulek a zkratek, existují-li, byl umístěn
%%% na začátku práce, místo na jejím konci.

\openright
\pagestyle{plain}
\setcounter{page}{1}
\tableofcontents

%%% Jednotlivé kapitoly práce jsou pro přehlednost uloženy v samostatných souborech
\chapter*{Introduction}
\addcontentsline{toc}{chapter}{Introduction}

\todo{Zatím jen nápady, přepsat, až bude zbytek hotový}



Problem of the coordination of a team of agents is widely discussed topic.
Designing of algorithms for such a coordination meets various difficulties 
including positional information interchanging, distribution of plan computation, 
communication reliability or system failures.

Monte-Carlo Tree Search is a best-first search algorithm which has
reached unprecedent results in playing a game of Go.
Accordingly to the proven qualities of the algorithms further works have
been dealing with it finding applications of the algorithm in various
domains.
Recently a work on team coordination using MCTS has been published by
Nguyen and Thawonmas \cite{Nguyen2011}.
MCTS is applied to controlling a team of ghosts in well-known Pacman
game in this paper.
Their computer player won the CEC 2011 Ms Pac-Man vs Ghost Team
Competition \cite{PacmanVsGhosts}.

\begin{itemize}
\item Nguien's ghosts runs simple instance of algorithm sharing tree
over all agents in team, no communication needed since joint moves are
calculated and passed to the environment
\item Our aim is to develop algorithms based on MCTS giving the autonomy
to agents what brings necessity of communication between agents
\item We consider the domain of Ms Pac-Man vs Ghost Team Competition
suitable for evaluation of the algorithms because the game of Ms Pac-Man
is well-known, frame
\item
\end{itemize}




\chapter{Monte-Carlo Tree Search}
\label{chap_mcts}

Mone-Carlo Tree Search (MCTS) is an iterative best-first search algorithm with stochastic positional
evaluation, anytime property and fast convergence. As \citeauthor{ChaslotPhd2010} summed up in 
\cite{ChaslotPhd2010}, MCTS was simultaneously developed
in three variants (\cites{Chaslot2006}{Coulom2006}{Kocsis2006}) in 2006. Specific variant used in
this thesis along with all important details and explanation of the properties is discussed in this
chapter.


\section{Algorithm Description}

\citeauthor{Chaslot2008} provides good description of the Monte-Carlo Tree Search algorithm in
\cite{Chaslot2008}. Variant of MCTS as well as the terminology used in this thesis are based mainly
on this paper.

Monte-Carlo Tree Search is iteratively building the search tree as depicted by picture
\emph{TODO}%\ref{fig_mcts_phases}
 and algorithm \ref{alg_mcts_loop}. Nodes of the tree contains at least two values - visit count
 saying how many random evaluations of positions in node's subtree have been executed and actual
 value which aggregates actual values from node's subtree (usually the average).

Each iteration of MCTS consists of four
phases - \emph{selection}, \emph{expansion}, \emph{simulation} and \emph{backpropagation}. During
the selection phase the 
algorithm passes through the tree to a particular leaf where better-evaluated but less-visited nodes
are preferred. Appropriate balance between these contradictory claims is main objective of this
phase. Once a leaf node is selected the expansion phase follows and all children nodes reachable
from the leaf node with one valid move are added to the leaf and one of these children is chosen.
The next phase, simulation (also called playout), plays a random game (or several ones) starting
in position defined by expanded node halting on some conditions (e.g. after certain moves are
played, the end of the game is reached). Results from the simulations are then backpropagated to all
expanded node's ancestors during the fourth phase (backpropagation). The phases of the algorithm are
further discussed in subsequent sections.

\begin{algorithm}
\DontPrintSemicolon
\caption{Monte-Carlo Tree Search\label{sec_mcts_description}}
\label{alg_mcts_loop}
$tree \leftarrow new\,McTree()$ \tcp*[h]{Initialize empty tree}\;
$root \leftarrow Root(tree)$\;
\While(\tcp*[h]{Main MCTS loop}){$EnoughTime()$}{
    $last\_node \leftarrow root$ \tcp*[h]{Phase 1: Selection}\;
    \While{$curr\_node \in tree$}{
        $last\_node \leftarrow curr\_node$\;
        $curr\_node \leftarrow Select(curr\_node)$\;
    }
    $last\_node \leftarrow Expand(last\_node)$ \tcp*[h]{Phase 2: Expansion}\;
    $reward \leftarrow Playout(last\_node)$ \tcp*[h]{Phase 3: Simulation}\;
    \While(\tcp*[h]{Phase 4: Backpropagation}){$curr\_node \in tree$}{
        $reward \leftarrow Backpropagate(curr\_node,\:reward)$\;
        $N_{curr\_node} \leftarrow N_{curr\_node}+1$ \tcp*[h]{Increase node's visit count}\;
        $curr\_node \leftarrow Parent(curr\_node)$\;
    }
}
\Return{$\argmax\limits_{n \in Children(root)}(N_{n})$} \tcp*[h]{Return most visited child}\;
\end{algorithm}


\subsection{Selection}

The process of selection consists of selection steps passing from a node to one of it's children.
Each such a step meets exploration-exploitation contradiction where exploitation tends to choose the
best node (one with the greatest current value) and exploration on the other side promotes
undiscovered ways in the tree. This problem is well-known as the Multi-Armed Bandit Problem (MAB)
\cites{Auer2002}{Kocsis2006}. 

\emph{XXX: Vadí (téměř) doslovný přepis definice?}

\begin{samepage}
\newtheorem*{defmab}{Definition}
\begin{defmab}[K-armed bandit problem ] 

Let us have independent random variables $X_{i,n}$ for $1 \le i \le K$ and $n \ge 1$. Each $i$ is
the index of a gambling machine and $X_{i,1}$, $X_{i,2}$,\ldots are identically distributed rewards
with unknown expected value $\mu_i$ yielded by successive plays of machine $i$. For the simplicity
the rewards are bounded to $[0,1]$.

A policy $A$ is an algorithm that chooses the next machine to play based on the sequence of
past plays and obtained rewards. Let $T_i(n)$ be the number of times machine $i$ has been played by
$A$ during the first $n$ plays and $I_i^A$ be the index of a machine played in nth play. Then the
regret of A after n plays is defined as

\begin{equation}
R_n^A = n \mu^* - \sum_{j=1}^K T_j(n) \mu_j \mathrm{,\;where}\;\mu^* \stackrel{\mathrm{def}}{=}
\max_{1 \le i \le K} \mu_i
\end{equation}

thus the regret $R_n^A$ is the loss caused by the policy not always playing the best machine.

K-armed bandit problem consists in finding optimal policy $A^*$ minimizing
expected regret $R_n^A$.

\end{defmab}
\end{samepage}

\emph{TODO: co citovat? Originální knížku z roku 85 nebo \cite{Auer2002}, kde se na ní odkazují?}
\emph{TODO: až se vyřeší citace, doplnit zmínku o tom, že optimální policy má regret  ~$O(log(n))$,
ale že je výpočetně náročná}

\Citeauthor{Auer2002} introduced computationaly effective optimal policy UCB1 (\cite{Auer2002}) having
regret bounded to $O(log(n))$ defined as:

\begin{equation}
 I_{UCB1}(n+1) = \argmax\limits_{i\in{1,\ldots,K}} \left({\bar{X}_{i,T_i(n)} 
+ \sqrt{{2 \log (n)} \over T_i(n)}}\right)
 \end{equation}.

Abbreviation UCB stands for \emph{Upper-Confidence Bound} and refers to value maximized by the
policy. UCB consists of average of previous rewards and a bias growing decreasing with number of the
machine's plays. In addition significance of the bias is growing with the total number of plays what
leads to increase of exploration.

UCB applied to Trees \cite{Kocsis2006} is derived from UCB using values $v_i$ and $n_i$ stored in
particular node $i$. In addition bias coefficient C is introduced in \cite{Chaslot2008} so the form of
UCT1 describing selection on node $p$ is:

\begin{equation}
I_{UCT1}^p = \argmax\limits_{i \in Children(j)} \left( v_i + C \sqrt{\log{n_p} \over{n_i}}\right)
\end{equation}

where 

 adopting a piece of
generalization in form 

\subsection{Expansion}


\subsection{Simulation}


\subsection{Backpropagation}



\section{Convergence to Minimax algorithm}
\label{sec_minimax_convergence}

\section{MCTS for Two-Player Games}
\emph{Discussion about simultaneous v. turn-based games and its convergence}

\section{Parallel Monte-Carlo Tree Search}








\chapter{Distributed Team Cooperation}
\emph{Extraction from the related chapters from \ref{MAS2008}}
\section{Distributed Constraints Optimization}

\section{Inter-Agent Communication}
\section{Conclusion}

\chapter{Evaluation of Distributed MCTS Algorithms}

\section{Ms.Pacman vs Ghosts Framework}

\section{Design Notes}

\section{Methodics}

\section{Results}


% Ukázka použití některých konstrukcí LateXu (odkomentujte, chcete-li)
% \include{example}

\chapter*{Conclusion}
\section*{Summary}
\section*{Future Work}
\addcontentsline{toc}{chapter}{Conclusion}


%%% Seznam použité literatury
%%% Seznam použité literatury je zpracován podle platných standardů. Povinnou citační
%%% normou pro diplomovou práci je ISO 690. Jména časopisů lze uvádět zkráceně, ale jen
%%% v kodifikované podobě. Všechny použité zdroje a prameny musí být řádně citovány.
%
%def\bibname{Bibliography}
%begin{thebibliography}{99}
%addcontentsline{toc}{chapter}{\bibname}

%\bibitem{mas2008} 
%{\sc Shoham} Yoav, {\sc Leyton-Brown} Kevin.
%\emph{Multiagent Systems: Algorithmic, Game-Theoretic, and Logical Foundations}
%Revision 1.1
%Cambridge University Press, 2008.

%end{thebibliography}
%\bibliography{bibliography}
%\bibliography{plain}
%\nocite{Chaslot2008}
%\nocite{Nguyen2011}
\nocite{Chaslot2008}
\nocite{MAS2008}
\nocite{Nguyen2011}
\nocite{Bouzy2007}

%\bibliographystyle{apacite}
%\bibliographystyle{plain}
%\bibliography{mendeley,my}
\printbibliography


%\include{bibl}

%%% Tabulky v diplomové práci, existují-li.
\chapwithtoc{List of Tables}

%%% Použité zkratky v diplomové práci, existují-li, včetně jejich vysvětlení.
\chapwithtoc{List of Abbreviations}

%%% Přílohy k diplomové práci, existují-li (různé dodatky jako výpisy programů,
%%% diagramy apod.). Každá příloha musí být alespoň jednou odkazována z vlastního
%%% textu práce. Přílohy se číslují.
\chapwithtoc{Attachments}

\openright
\end{document}
