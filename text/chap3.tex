\chapter{Distributed MCTS Algorithms for Cooperating Teams}
\label{chap_dmcts_design}

\section{Motivation and Problem Definition}

\todo{Upravit podle toho, co bude skutečně v kapitole \ref{chap_mas}}

We have already reviewed the Monte-Carlo Tree Search algorithm and its parallelization in
 Chapter \ref{chap_mcts} and distributed coordination algorithms and in Chapter \ref{chap_mas}.
 The purpose of this chapter is design of distributed coordination algorithms based on
 Monte-Carlo Tree Search. Presented algorithms will work with MCTS tree kept indepentently in
 each cooperating agent. The tree will contain full information about actions done by all
 cooperating agents and also agents from opponent teams. This approach is supposed to be
 appropriate for games with small number of players but with increasing of the number of
 players, the branching factor grows and keeping full MCTS tree turns to be ineffective. The
 tree used in distributed algorithms along with general description of common parts of the
 algorithms are described in section \ref{sec_dmcts_common}. Before we will propose the
 algorithms description, it is necessary to briefly discuss measures used for comparison of
 distributed algorithms in Section \ref{sec_measures_distributed}. \todo{Popis full MCTS stromu
 - zde nebo už v kapitole 1?}



\section{Comparison Measures}
\label{sec_measures_distributed}

Measures suitable for plain Monte-Carlo Tree Search and parallel
Monte-Carlo Tree Search have been already described in Section \ref{sec_measures_parallel}.
We will show that these
measures are also suitable for comparison of distributed MCTS algorithms for games with team
of cooperating agents. In addition, we will compare the amount of communication needed by the
algorithms and the robustness of the algorithms against communication failures. For such
comparisons, we will evaluate strength measures depending on the amount of communication and
its robustness. The amount of communication will be simply the total length of messages
exchanged. Some environments (e.g. radio transmissions or Ethernet over coax) provide the 
possibility of broadcasting messages for the cost of passing single message whereas others 
don't so the cost of broadcasted message equals to the cost of separate messages to all
receivers. We will distinguish between these
environments since some of the algorithms advantage from cheap message broadcasting.

We have described two classes of measures for parallel MCTS, \emph{strength-} and
\emph{simulations-per second-}based measures. Former one works with score obtained at the end
of a game and latter one counts number of MCTS iterations handled per time unit. Details of
these measures are discussed in Section \ref{sec_measures_parallel}. Obviously we can use
simulations-per-second measures, distribution is not obstacle, simulations are still performed
and these measures show the computational costs of distribution of computation. Similar is the
case of strength measures where the only difference is that agents decide the actions
independently but from outer point of view the joint action is the same as if plain or parallel
MCTS calculate action. Strength measures say how strong is team guided by an algorithm.
\todo{Jak budou simulovány communicatin failures?}

\section{Proposed Algorithms}

\todo{Algoritmy s plným stromem x algoritmy inspirované DCOPy...}

\subsection{Backbone of the Distributed Algorithms}
\label{sec_dmcts_common}

MCTS-based agents will iteratively build MCTS tree exactly as plain MCTS algorithm does. In
addition before certain set of MCTS iterations, the agent receives messages from its team-mates
and performs appropriate actions. Similarly after the set of MCTS iterations, the agent
broadcasts messages to the others. The communicaton between agents is the point differencing
particular distributed algorithms. We call the agent building full MCTS tree 





\subsection{Independent Agents}




\subsection{Joint Action Exchange}
\subsection{Root Exchange}
\subsection{Simulation Results Passing}
\subsection{Tree Cut Exchange}
